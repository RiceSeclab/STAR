\label{sec:threat}

To evaluate the design and engineering of \projname, it's helpful to
have a threat model in mind. The obvious place to start would be
VoteBox~\cite{sandler08votebox}, which is closely related to
\projname. The original VoteBox authors did not state a concise threat
model, but considered several kinds of threats and security design
goals:

\begin{compactdesc}
\item[Software independence]
    \projname, like VoteBox, should be able to produce a proof of the
    correctness of an election that does not require any statement
    about the correctness of the software used in \projname. VoteBox
    achieved this through end-to-end cryptographic means. \projname
    uses similar cryptography and adds a risk-limiting audit that can
    verify the correspondence between \projname printed ballot records
    and their electronic counterparts, adding a degree of flexibility
    if the cryptography cannot prove an exact correspondence to determine exactly what went wrong. 

\item[Reduced trusted computing base]
  \projname, like VoteBox or any other software artifact, would
  benefit from having simpler code and less of it. This makes it
  easier to verify and less likely to have bugs. Software independence
  means that \projname's software is not required for {\em
    correctness} of the election outcome, but it does help defeat
  attacks which could disable the system, destroy records, or
  otherwise cause grief to election officials running \projname.
  VoteBox specifies that it uses pre-rendered user interfaces~\cite{yee07pvote,yee06prui}.
  \projname should probably use this technique as well. 

\item[Robustness against data loss]
  \projname, like VoteBox, specifies that vote records be stored on
  every voting terminal in the local polling place, using
  tamper-evident logging techniques. \projname adds a printed
  ballot record, stored in a ballot box. VoteBox went a step further
  by considering the real-time one-way transfer of vote records out of
  the polling place, across the Internet, to a central election
  headquarters. While \projname could add this in the future, it's not part of the
  initial design.

\item[Mega attacks]
  In the VoteBox paper, the authors considered a variety of attacks with highly capable attackers. Such attackers might run a concurrent election on parallel equipment, in an attempt to substitute the results for genuine votes. Other attackers might mount a ``booth capture'' attack, wherein armed gunmen take over a polling place and cast votes as fast as possible until the police arrive. These attacks, needless to say, are well within the ability of \projname's cryptographic and risk-limiting infrastructure to detect. The best such attackers can hope to do is, in effect, mount a denial of service attack against the election. Attackers with that as their goal can arrive at much simpler approaches and \projname has relatively little it can offer beyond any other election system in this regard.
\end{compactdesc}

A full consideration of threats to \projname and their corresponding countermeasures or mitigations would be far too long to fit in this paper. Instead, we focus on several areas where \projname differs from other e2e voting systems in the literature.

\subsection{Coercion}
\label{sec:coercion}
In designing \projname, we made several explicit decisions regarding how much to complicate the protocol and impede the voter experience in order to mitigate known coercion threats.  Specifically, one known threat is that a voter is instructed to create a ballot in a particular way but to then execute a decision to cast or spoil the ballot according to some stimulus received after the ballot has been completed and the receipt has been generated.  The stimulus could come, for example, from subtle motions by a coercer in the poll site, the vibration of a cell phone in silent mode, or some of the (unpredictable) data that is printed on the voter’s receipt.  Some prior protocols have required that the receipt, although committed to by the voting device, not be visible to the voter until after a cast or spoil decision has been made (perhaps by printing the receipt face down behind a glass barrier) and configuring poll sites so that voters cannot see or be seen by members of the public until after they have completed all steps.  We could insist on similar measures here, but in an era where cell phones with video recording capabilities are ubiquitous and eyeglasses with embedded video cameras can easily be purchased, it seems unwise to {\em require} elaborate measures which mitigate some coercion threats but leave others unaddressed.

\subsubsection{Chain voting}
A similar threat of ``chain voting'' is possible with this system wherein a voter early in the day is instructed to neither cast nor spoil a ballot but to instead leave the poll site with a printed ballot completed in a specified way.  This completed ballot is delivered to a coercer who will then give this ballot to the next voter with instructions to cast the ballot and return with a new printed ballot---again completed as specified.  Chain voting can be mitigated by instituting timeouts which automatically spoil ballots that have not been cast within a fixed period after having been printed. We also expect to have procedures in place to prevent voters from accidentally leaving poll sites with printed ballots. We note that the timeout period need only cover the time we expect will be required for a voter to cross the room with a printed ballot and place it in the box, allowing for a relatively tight time bound, probably less than 5 minutes, although we'd need to run this in practice to understand the distribution of times that might happen in the real world.

(We note that traditional paper ballots sometimes include a perforated header section which includes a serial number. A poll worker keeps one copy of this number and verifies that the ballot a voter wishes to cast matches the expected serial number. If so, the serial number is then detached from the ballot and deposited in the box. \projname could support this, but we believe it would damage \projname's usability. The timeout mechanism seems like an adequate mitigation.)

We do, however, take measures to prevent wholesale coercion attacks such as those that may be enabled by pattern voting.  For instance, The SOBA audit process is explicitly designed to prevent pattern-voting attacks; and the high assurances in the accuracy of the tally are acheived without ever publishing the full set of raw ballots.

% Similarly, \projname makes no particular effect to defeat pattern voting (a.k.a., the ``Italian attack,'' wherein uninteresting races are used to encode a voter-unique serial number). We assume that attackers will not have unrestricted access to the printed plaintext ballots, and we assume that attackers with full access to the public bulletin board would be unable to decrypt individual ballots. They would only be able to verify the decryption of homomorphic tallies.

An interesting concern is that our paper ballots have data on them to connect them to electronic ballot records from the voting terminals and judge's console. The very data that links a paper ballot to an electronic, encrypted ballot creates a potential vulnerability. Since some individual paper ballot summaries will be selected for post-election audit and made public at that time, we are careful to not include any data on the voter's take-home receipt which can be associated with the corresonding paper ballot summary.

% dwallach: I'm not sure I believe the statement below, which is my attempt to rewrite the one below that. For now, I'm leaving it blank.

% Measures are taken to avoid introducing new means of coercion.  For example, the required identifiers on printed ballots are dissociated from take-home receipts by not including any publicly-linkable data.  Thus, even if an attacker had access to the ballot box and could peruse all the printed ballots, there would be no way to associate any given paper ballot with the voter's take-home receipt.

%complete copies of all paper ballots are made public, there will be no loss of privacy or coercion opportunities introduced.

\subsubsection{Absentee and provisional ballots}
There are several methods available for incorporating ballots which are not cast within the \projname system, such as absentee and provisional ballots.  The simplest approach is to completely segregate votes and tallies, but this has several disadvantages, including a reduction in voter privacy and much lower assurance of the accuracy of the combined tally.

It may be possible to eliminate all ``external'' votes by providing electronic means for capturing provisional and remote ballots.  However, for the initial design of the \projname system, we have chosen to avoid this complexity.  Instead, we ask that voting officials receive external votes and enter them into the \projname system as a proxy for voters.  While this still does not allow remote voters to audit their own ballots, the privacy-preserving risk-limiting audit step is still able to detect any substantive deviations between the paper records of external voters and their electronically recorded ballots.  This provides more supporting evidence of the veracity of the outcome without reducing voter privacy.


\subsection{Further analysis}

If we wished to conduct a more in-depth threat modeling exercise, one place to begin would be
the threat model developed by the California Top To Bottom Review's source code audit teams (see, e.g., \cite{ca-hart-src07}). They considered different levels of attacker access, ranging from voters to election officials. They also considered different attacker motives (disrupt elections, steal votes, coerce voters) and different attack outcomes (detectable vs. undetectable, recoverable vs. unrecoverable, prevention vs. detection, wholesale vs. retail, and casual vs. sophisticated). A complete consideration of \projname against all these criteria would take far more space than is available in this venue. Instead, we now focus on where \projname advances the state of the art in these areas.

Most notably, \projname's combination of end-to-end cryptography with risk-limiting audits of paper ballots is a game changer, in terms of thwarting attackers who might want to disrupt elections. Unlike paperless systems, \projname has the ability to fall back to the paper records, with efficient processes to detect when inconsistencies exist that would require this. This radically improves \projname's recoverability from extreme failures.

Similarly, while \projname is ``software independent,'' we must concern ourselves with software tampering that does not change any of the cryptographic computations, but instead causes the \projname machines to silently record everything the voter does. This threat cannot be mitigated by better cryptography or ballot auditing. The only likely solution is some sort of trusted platform management (TPM), where the hardware will refused to run third-party code (more discussion on this appears in Section~\ref{sec:sw-hw-eng}).

Lastly, we consider a threat that only arises in e2e systems: presentation of a fraudulent voting receipt. Consider the case where a voter may spoil her ballot and take it home to verify against the public bulletin board. A malicious voter with access to similar printers could produce a seemingly legitimate ballot for which there is no correspondence on the public bulletin board, thus ``proving'' that the election system lost a record. Similar defaming attacks could be made by forging the receipt that a voter can take home after casting a ballot. For \projname, we have considered a number of mitigations against these attacks, ranging from cryptographic (having the voting terminals compute a digital signature, with protected key material) to procedural (e.g., watermarking the paper or having poll workers physically sign spoiled ballots). Real \projname deployments will inevitably use one or more of these mitigations.

%%% Local Variables: 
%%% mode: latex
%%% TeX-master: "star"
%%% End: 
