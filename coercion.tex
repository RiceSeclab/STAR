\label{sec:coercion}

In designing \projname, we made several explicit decisions regarding how much to complicate the protocol and impede the voter experience in order to mitigate known coercion threats.  Specifically, one known threat is that a voter is instructed to create a ballot in a particular way but to then execute a decision to cast or spoil the ballot according to some stimulus received after the ballot has been completed and the receipt has been generated.  The stimulus could come, for example, from subtle motions by a coercer in the poll site, the vibration of a cell phone in silent mode, or some of the (unpredictable) data that is printed on the voter’s receipt.  Some prior protocols have required that the receipt, although committed to by the voting device, not be visible to the voter until after a cast or spoil decision has been made (perhaps by printing the receipt face down behind a glass barrier) and configuring poll sites so that voters cannot see or be seen by members of the public until after they have completed all steps.  We could insist on similar measures here, but in an era where cell phones with video recording capabilities are ubiquitous and eyeglasses with embedded video cameras can easily be purchased, it seems unwise to {\em require} elaborate measures which mitigate some coercion threats but leave others unaddressed.

\subsection{Chain Voting}
A similar threat of ``chain voting'' is possible with this system wherein a voter early in the day is instructed to neither cast nor spoil a ballot but to instead leave the poll site with a printed ballot completed in a specified way.  This completed ballot is delivered to a coercer who will then give this ballot to the next voter with instructions to cast the ballot and return with a new printed ballot---again completed as specified.  Chain voting can be mitigated by instituting time-outs which automatically spoil ballots that have not been cast within a fixed period after their production and by attempting to prevent voters from leaving poll sites with printed ballots, but, beyond simple mitigations, we require no additional steps to make chain voting impossible. 

(Traditional paper ballots sometimes include a perforated header section which includes a serial number. A poll worker keeps one copy of this number and verifies that the ballot a voter wishes to cast matches the expected serial number. If so, the serial number is then detached from the ballot and deposited in the box. \projname could support this, but we believe it would damage \projname's usability. The timeout mechanism seems like an adequate mitigation.)

We do, however, take measures to prevent wholesale coercion attacks such as those that may be enabled by pattern voting.  For instance, The SOBA audit process is explicitly designed to prevent pattern-voting attacks; and the high assurances in the accuracy of the tally are acheived without ever publishing the full set of raw ballots.

% Similarly, \projname makes no particular effect to defeat pattern voting (a.k.a., the ``Italian attack,'' wherein uninteresting races are used to encode a voter-unique serial number). We assume that attackers will not have unrestricted access to the printed plaintext ballots, and we assume that attackers with full access to the public bulletin board would be unable to decrypt individual ballots. They would only be able to verify the decryption of homomorphic tallies.

An interesting concern is that our paper ballots have data on them to connect them to electronic ballot records from the voting terminals and judge's console. The very data that links a paper ballot to an electronic, encrypted ballot creates a potential vulnerability. Since some individual paper ballot summaries will be selected for post-election audit and made public at that time, we are careful to not include any data on the voter's take-hme receipt which can be associated with the corresonding paper balot summary.

% dwallach: I'm not sure I believe the statement below, which is my attempt to rewrite the one below that. For now, I'm leaving it blank.

% Measures are taken to avoid introducing new means of coercion.  For example, the required identifiers on printed ballots are dissociated from take-home receipts by not including any publically-linkable data.  Thus, even if an attacker had access to the ballot box and could peruse all the printed ballots, there would be no way to associate any given paper ballot with the voter's take-home receipt.

%complete copies of all paper ballots are made public, there will be no loss of privacy or coercion opportunities introduced.

\subsection{Absentee and Provisional Ballots}
There are several methods available for incorporating ballots which are not cast within the \projname system, such as absentee and provisional ballots.  The simplest approach is to completely segregate votes and tallies, but this has several disadvantages, including a reduction in voter privacy and much lower assurance of the accuracy of the combined tally.

It may be possible to eliminate all ``external'' votes by providing electronic means for capturing provisional and remote ballots.  However, for the initial design of the \projname system, we have chosen to avoid this complexity.  Instead, we ask that voting officials receive external votes and enter them into the \projname system as a proxy for voters.  While this still does not allow remote voters to audit their own ballots, the privacy-preserving risk-limiting audit step is still able to detect any substantive deviations between the paper records of external voters and their electronically recorded ballots.  This provides more supporting evidence of the veracity of the outcome without reducing voter privacy.



%%% Local Variables: 
%%% mode: latex
%%% TeX-master: "star"
%%% End: 
