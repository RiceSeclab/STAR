\label{sec:intro}

A decade ago, DRE voting systems came with a promise of improvement. By having a
computer mediating the user's voting experience, they could ostensibly
improve usability through summary screens, as well as supporting a
variety of accessibility features including enlarged text, audio
output, and specialized input devices. They also promised to improve
the life of the election administrator, yielding quick tallies without
any of the ambiguities that come along with hand-marked paper ballots.
And, of course, they were promised to be secure and reliable, tested
and certified. In practice, much of this was wishful thinking.

Many of the current DRE voting systems experienced their biggest sales
volume following the demonstrated failures of punch card voting
systems in Florida in the 2000 presidential election. The subsequent Help America Vote Act
provided a one-time injection of funds that made these purchases
possible. Now, a decade later, these machines are beginning to reach the end of
their service lifetimes. 

Last year, the election administration office of a county that was an
early adopter of these DRE systems reached the conclusion that current
systems on the market were inadequate for their need to replace their
end-of-life DRE systems. They were also unhappy with the
current-generation precinct-based optical scanned paper ballot systems
for a variety of reasons. In particular, hand-marked paper ballots
open the door to ambiguous voter intent, which this county unhappily
had to deal with in its earlier centrally-tabulated optical scan
system. They didn't want to go back. Likewise, with early
voting and election day vote centers, allowing any voter to arrive at
any location and be given the proper ballot style, pre-printed paper ballots would
be a management nightmare, and ballot-on-demand printing systems
require laser printers that cannot run all day on battery backup
systems\footnote{A laser printer may consume as much as 1000 watts
  while printing. A reasonably good UPS, weighing 26~kg, can provide
  that much power for only ten minutes. Since a printer must take time
  to warm up for each page when printed one-off (perhaps 10
  seconds total per page), as few as 60 ballots could be printed before the
  battery would be exhausted.}.

A group of academic experts in voting systems was collected for the
purpose of attempting to specify a replacement system, in sufficient
detail that bids could be solicited from manufacturers to implement
the system. Our group included experts in cryptography, auditing, and
usability, leading to some interesting challenges and questions. We
were charged with several basic constraints. The user experience must
resemble current DRE systems, but there should be a tangible
voter-verifiable paper ballot, printed by the machine and deposited by
the voter in a physical ballot box. This would allow for fast machine
tallies and statistical audits to ensure their equivalence to the
paper records. We were free to recommend sophisticated end-to-end
cryptographic methods, privacy-preserving risk-limited auditing
methods, and pretty much whatever else we felt was beneficial, given
these constraints. Of course, reducing cost was also desirable as was
anything that might reduce the burden for poll workers or voters. The
issue of federal or state election certification, for the purpose of
this exercise, was considered out of scope. (Yes, it's a real
challenge, but it wasn't our challenge.)

%%% Local Variables: 
%%% mode: latex
%%% TeX-master: "star"
%%% End: 
