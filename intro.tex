\label{sec:intro}

A decade ago, DRE voting systems came with a promise of improvement. By having a
computer mediating the user's voting experience, they could ostensibly
improve usability through summary screens and a
variety of accessibility features including enlarged text, audio
output, and specialized input devices. They also promised to improve
the life of the election administrator, yielding quick, accurate tallies without
any of the ambiguities that come along with hand-marked paper ballots.
And, of course, they were promised to be secure and reliable, tested
and certified. In practice, much of this was wishful thinking.

Many current DRE voting systems experienced their biggest sales
volume following the demonstrated failures of punch card voting
systems in Florida in the 2000 presidential election. The subsequent Help America Vote Act
provided a one-time injection of funds that made these purchases
possible. Now, a decade later, these machines are near the end of
their service lifetimes. 

Last year, the election administration office of a large county, an
early adopter of these DRE systems, concluded that current
systems on the market were inadequate for their need to replace their
end-of-life DRE systems. They were also unhappy with the
current-generation precinct-based optical scanned paper ballot
systems.
for a variety of reasons. In particular, hand-marked paper ballots
open the door to ambiguous voter intent, which this county unhappily
had to deal with in its previous centrally-tabulated optical scan
system. They didn't want to go back.

Fundamentally, we were given a number of constraints in the design of \projname:
\begin{compactdesc}
\item[DRE-style UI] 
The county had previously worked with centrally tabulated,
    hand-marked paper ballots prior to their DRE experience, and they
    had no desire to deal with ambiguous voter intent. DRE-style
    systems were also preferred for their ability to offer facilities for
    disabled voters.

\item[Printed paper ballot summaries]
    While the DRE-style UI was
    desired for ballot entry, printed ballots were desired for their
    obvious security benefits, having been verified by voters, and redundancy against failures in the
    electronic system. In order to save on paper, the county wished to
    only print a list of the selected candidates, analogous to the
    summary screens on many current-generation DRE systems.

\item[All-day battery life]
    Power failures happen. Current-generation DRE systems have
    batteries that can last for hours. The new system must also be
    able to operate for hours without mains power.

\item[Early voting and election-day vote centers]
    This county supports two weeks of early voting, where any voter
    may go to any of more than 20 locations across the county. Also,
    on Election Day, any voter may go to any local polling
    place. Voters apparently love this flexibility.

\item[COTS hardware]
    Current DRE systems are surprisingly expensive. The county wants
    to use commercially available, off-the-shelf equipment, whenever
    possible, to reduce costs and shorten upgrade cycles.

\item[Long ballots]
    While voters in many countries select their candidate for member
    of parliament and that's it, in our county, voters regularly face
    100 or more contests, electing federal, state, and regional
    officials as well as selecting judges and voting on propositions
    and constitutional amendments. \projname must support very long
    ballots.
\end{compactdesc}

These constraints interact in surprising ways. Pre-printed paper
ballots, even if we relaxed the requirement for the DRE-style user
interface, would still be unacceptable for vote centers, which may
need to support hundreds or thousands of distinct ballot
styles. Likewise, the requirement to run all-day on battery backup
eliminates the possibility of using laser printers, which consume too
much power\footnote{A laser printer may consume as much as 1000 watts
  while printing. A reasonably good UPS, weighing 26~kg, can provide
  that much power for only ten minutes. Since a printer must take time
  to warm up for each page when printed one-off (perhaps 10
  seconds total per page), as few as 60 ballots could be printed before the
  battery would be exhausted.}.

We were charged with using the latest advances in human factors,
end-to-end cryptography, and statistical auditing techniques, while
keeping costs down and satisfying so many challenging constraints.
We want to generate quick, verifiable tallies when the election is
over, yet enabling a variety of audit mechanisms (some
voter-verifiable, others conducted by auditors with additional
privileges). 

We were notably not required to worry about Federal and State
certification. Of course, for \projname to go into production, these
challenges would need to be addressed, but at least for now, our focus
has been on designing the best possible voting system given these
constraints.

%%% Local Variables: 
%%% mode: latex
%%% TeX-master: "star"
%%% End: 
