\label{sec:intro}

A decade ago, DRE voting systems promised to improve many aspects of voting. 
By having a
computer mediating the user's voting experience, they could ostensibly
improve usability through summary screens and a
variety of accessibility features including enlarged text, audio
output, and specialized input devices. They also promised to improve
the life of the election administrator, yielding quick, accurate tallies without
any of the ambiguities that come along with hand-marked paper ballots.
And, of course, they were promised to be secure and reliable, tested
and certified. In practice, much of this was wishful thinking.

Many current DRE voting systems experienced their biggest sales
volume after the failures of punch card voting
systems in Florida in the 2000 presidential election. 
The subsequent Help America Vote Act
provided a one-time injection of funds that made these purchases
possible. Now, a decade later, these machines are near the end of
their service lifetimes. 

Last year, the election administration office of a large county
whose current DRE systems are approaching end-of-life
concluded that no system on the 
market---DRE or optical scan---meets their needs. 
This county, an early adopter of DREs, prefers to avoid 
hand-marked paper ballots because they
open the door to ambiguous voter intent, a source of
frustration in its previous centrally-tabulated optical scan
system. 
They didn't want to go back.

The county's needs and preferences 
impose several significant constraints on the design of \projname:
\begin{compactdesc}
\item[DRE-style UI] 
    Hand-marked ballots are not to be used, for the reason above.
    DRE-style
    systems were also preferred for their ability to offer facilities for
    disabled voters.

\item[Printed paper ballot summaries]
    While the DRE-style UI was
    desired for entering selections, printed ballots were desired for their
    security benefits, verifiability by voters, and redundancy against failures in the
    electronic system. In order to save on paper, the county wished to
    only print a list of each voter's selections, analogous to the
    summary screens on many current-generation DRE systems.

\item[All-day battery life]
    Power failures happen. Current-generation DRE systems have
    batteries that can last for hours. The new system must also be
    able to operate for hours without mains power.

\item[Early voting and election-day vote centers]
    This county supports two weeks of early voting, where any voter
    may vote in any of more than 20~locations. Also,
    on Election Day, any voter may go to any local polling
    place. Voters report loving this flexibility.

\item[COTS hardware]
    Current DRE systems are surprisingly expensive. The county wants
    to use commercially available, off-the-shelf equipment, whenever
    possible, to reduce costs and shorten upgrade cycles.
    That is, ``office equipment''  rather than ``election
    equipment'' should be used where possible.

\item[Long ballots]
    While voters in many countries only select a candidate for member
    of parliament, in the U.S., voters regularly face
    100 or more contests for federal, state, and regional
    offices; judges; propositions;
    and constitutional amendments. \projname must support very long
    ballots.
\end{compactdesc}

These constraints interact in surprising ways. 
Even if the county did not have a strong preference for a DRE-like UI, 
pre-printed paper
ballots are inefficient for vote centers, which may
need to support hundreds or thousands of distinct ballot
styles. Likewise, the requirement to run all-day on battery backup
eliminates the possibility of using laser printers, which consume too
much power.\footnote{%
   A laser printer may consume as much as 1000~watts
  while printing. A reasonably good UPS, weighing 26~kg, can provide
  that much power for only ten minutes. Since a printer must 
  warm up for each page when printed one-off (perhaps 10
  seconds per page), the battery might be exhausted by printing
  as few as 60 ballots.
 }

We were charged with using the latest advances in human factors,
end-to-end cryptography, and statistical auditing techniques, while
keeping costs down and satisfying many challenging constraints.
We want to generate quick, verifiable tallies when the election is
over, yet incorporate a variety of audit mechanisms (some
voter-verifiable, others conducted by auditors with additional
privileges). 

We were notably not required to worry about Federal and State
certification. Of course, for \projname to go into production, these
challenges need to be addressed, but at least for now, our focus
has been on designing the best possible voting system given these
constraints.

%%% Local Variables: 
%%% mode: latex
%%% TeX-master: "star"
%%% End: 
