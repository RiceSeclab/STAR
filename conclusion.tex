\label{sec:future}
\label{sec:conclusion}

In many ways, \projname is a straightforward evolution from existing
commercial voting systems, like the Hart InterCivic eSlate, mixing in
advanced cryptography, software engineering, usability, and auditing
techniques from the research literature in a way that will go largely
unnoticed by most voters, but that have huge impact on the
reliability, accuracy, fraud-resistance, and transparency of
elections. Due to space constraints, this document doesn't 
mention many pragmatic features that our election
administration colleagues have specified based on their experience
running prior elections. Clearly, we're long overdue for election
systems engineered with all the knowledge we now have available.

\projname also opens the door to a variety of interesting future
directions. 
For example, while \projname is intended to service any
given county as an island unto itself, there's no reason why it cannot
also support {\em remote voting}, where ballot definitions could be
transmitted anywhere a voter wishes to vote, and results sent back
home. 
By virtue of \projname's cryptographic mechanisms, such a remote
vote is really no different than a local provisional vote and can be
resolved in a similar fashion, preserving the anonymity of the
voter. (A variation on this idea was earlier proposed as the RemoteBox
extension~\cite{remotebox08} to VoteBox~\cite{sandler08votebox}.)
This could have important ramifications for overseas and military
voters with access to a suitable impromptu polling place, e.g., on a
military base or in a consular office.

(We do not want to suggest that \projname
would be suitable for {\em Internet} voting. Using computers of
unknown provenance, with inevitable malware infections, and
without any systematic way to prevent voter bribery or coercion,
would be a foolhardy way to cast ballots.)

\projname anticipates the possibility that voting machine
hardware might be nothing more than commodity computers running custom
software. It remains unclear whether off-the-shelf computers can be
procured to satisfy all the requirements of voting systems (e.g.,
long-term storage without necessarily having any climate control, or
having enough battery life to last for a full day of usage), but
perhaps such configurations might be possible.


%%% Local Variables: 
%%% mode: latex
%%% TeX-master: "star"
%%% End: 
