\label{sec:crypto}

\COP{I am focusing here on the crypto-related aspects only.}

\paragraph{The core elements}
\label{sec:crypto-core}

From the perspective of election officials, the first new element in
the election regiment is the cryptographic key generation process. A
set of election trustees is designated as key holders and a threshold
number is fixed. The functional effect is that if there are $n$
election trustees and the threshold value is $t$, then any $t$ of the
$n$ trustees can complete the election – even if the remaining $n-t$
are unavailable. This threshold mechanism provides robustness while
preventing any fewer than $t$ of the trustees from performing election
functions which might compromise voter privacy. Depending on the
availability of the election trustees and on the desired amount of
robustness, a choice between two procedures can be made: either the
trustees are available for a general two-steps key generation
procedure, or a simpler one-step procedure can be adopted if $t=n$.

At the end of the key generation procedure, the trustees each hold a
private key share that does not contain any information on the full
private key, and the unique public key $K$ corresponding to those
shares is published.

During the polling phase, the ballot marking devices encrypt the votes
each voter using the public key $K$ and an additively homomorphic
encryption scheme (e.g., \elgamal). This encryption procedure is
randomized in order to make sure that two votes for the same
candidates result in ciphertexts that look independent to any
observer. 

A short hash value of each ciphertext is also computed, e.g., by
truncating the output of the SHA256 hash function.  This hash provides
a unique fingerprint of the ballot, which is provided to the voter as
part of his vote receipt. All the ciphertexts and hashes that are
computed are posted on a publicly accessible web page, either
immediately or as soon as the polls are closed. This web page is
digitally signed by the election office using a traditional signature
key (not the key generated by the trustees).

By posting all those encrypted ballots and hashes, every voter gets
the possibility to verify that his ballot was properly recorded. As a
second benefit, this web page makes it possible, thanks to the
homomorphic property of the encryption scheme, to publicly aggregate
all the encrypted ballots into an encryption of their sum, that is, an
encryption of the outcome of the election.

At the end of the election, $t$ trustees take part to the decryption
of this outcome. This procedure is simple and efficient: the trustees
can perform it in one step, and the number of ciphertexts to be
decrypted is independent of the number of voters. We note that the
tally is computed without decrypting any single individual ballot.

The elements we just described make the core of the workflow and are
sufficient to compute an election tally while preserving the privacy
of the votes. We now explain various ways in which this simple
workflow is hardened in order to make sure that the tally is also
correct. All the techniques that follow enable the verification of
different aspects of the ballot preparation and casting.

\paragraph{Hardening encryption}
\label{sec:hardening-encryption}
Since the tally does not involve the decryption of any individual
ballot, and since the audit procedure relies on the fact that all
counted ballots are properly formed, it is crucial to make sure that
all the encrypted ballots that are added correspond to valid votes.
This is achieved by requiring the ballot marking devices to compute,
together with the encryption of the votes, a non-interactive
zero-knowledge (NIZK) proof that each ballot is well-formed. Such a
proof guarantees that a ciphertext encrypts a valid vote and does not
leak any other information about the content of the vote. As a side
benefit, this proof can be designed to make the ballots non-malleable,
which provides an easy technique to prevent the replay of old ballots
(i.e., reject duplicates).

\paragraph{Hardening decryption}
\label{sec:hardening-decryption}
Making sure that the encrypted ballots are valid is not enough: we
also need to make sure that the tally is correctly decrypted as a
function of those encrypted ballots: otherwise, malicious trustees (or
trustees using corrupted devices) could publish an outcome that does
not correspond to these ballots. As a result, we require the trustees
to provide evidence of the correctness of the decryption operations
that they perform.  Again, NIZK proofs are a standard tool to achieve
this.

\paragraph{Hardening the timeline}
\label{sec:hard-timeline}

The procedures described above prevent malfunctioning or corrupted
marking devices or trustees to falsify individual ballots or decryption
operations.

The detection of manipulation of encrypted ballots can be more
effective by linking ballots with each other, using hash chaining.  To
this purpose, each ballot marking device is seeded, at the beginning
of the election, with a public start value $z_0$ that includes a
unique identifier for the election. This unique identifier should be
hard to predict and chosen as close as possible from the beginning of
the polls in order to reduce the time frame during which a malicious
party could prepare an attack.

This $z_0$ seed is then used as follows: as soon as a ballot marking
device with identifier $id_m$ computes an encrypted ballot $b_i$, it
computes a hash $z_i := H(b_i \| z_{i-1} \| id_m)$ and broadcasts that
value to all other machines on the network. Then, as soon as the polls
are over, the latest $z$ value is digitally signed and made public.

As a result of this procedure, any forgotten or erased ballot will
invalidate the hash chain. In a similar way, it is not possible for a
corrupted marking device to go in the past and to build a full hash
chain that would be consistent, since this hash chain is broadcast as
it is created.

\paragraph{Hardening the link between paper and electronic ballots}
\label{sec:hard-link-betw}

\COP{To be clarified.}

\paragraph{Summing up} The elements described above lead to the
following workflow. \COP{Sum up, explaining how all elements are
  computed and in what order.}
% \begin{compactenum}
% \item The trustees choose a threshold $t$ and run a the corresponding
%   key generation procedure. At the end of this procedure, each trustee
%   holds a private key and a unique public key $K$ is published.
% \item At the beginning of the election day, a public start value $z_0$
%   is chosen that uniquely identifies the election and voting
%   office. All ballot marking devices are initialized with the election
%   specifics (ballot format, \dots), with the key $K$ and with the seed $z_0$.
% \item After the completion of the selection by each voter, the ballot
%   marking device proceeds as follows.
%   \begin{compactenum}[a.]
%   \item It computes an encryption $b_i$ of the voter's selection using
%     the public key $K$, together with a ballot validity proof
%     $pb_i$. These elements are hashed in order to produce a ballot
%     fingerprint $f_i = H(b_i, pb_i, z_{i-1}, m)$.
%   \item It produces paper printouts of two items. The first is the
%     paper ballot which consists of the selections made by the voter
%     and $f_i$. The second is a receipt that consists of an
%     identification number $m$ for the ballot marking device, the date
%     and time of the vote, and a truncated SHA-256 hash of the \elgamal
%     encryption of the voter’s selections together with the previous
%     hash value.

%   \end{compactenum}
% \end{compactenum}


%%% Local Variables: 
%%% mode: latex
%%% TeX-master: "star"
%%% End: 
