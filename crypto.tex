\label{sec:crypto}


% Notations: 
% \begin{compactitem}
% \item $cv = E_K(v)$ encryption of vote $v$ with proof $p_c$ 
% \item $cs = E_K(s)$ serial number $s$ encryption
% \item $c = (cv, cs)$
% \item $z_i$ hash chaining
% \item $f_i$ hash of encrypted ciphertext
% \item $\pi$ permutation used in verifiable shuffle
% \item $m$ voting machine id
% \end{compactitem}

\paragraph{The core elements}
\label{sec:crypto-core}

% From the perspective of election officials, the first new element in
% the election regiment is the cryptographic key generation process. A
% set of election trustees is designated as key holders and a threshold
% number is fixed. The functional effect is that if there are $n$
% election trustees and the threshold value is $t$, then any $t$ of the
% $n$ trustees can complete the election – even if the remaining $n-t$
% are unavailable. This threshold mechanism provides robustness while
% preventing any fewer than $t$ of the trustees from performing election
% functions which might compromise voter privacy. Depending on the
% availability of the election trustees and on the desired amount of
% robustness, a choice between two procedures can be made: either the
% trustees are available for a general two-steps key generation
% procedure, or a simpler one-step procedure can be adopted if $t=n$.

\newcommand{\Ext}{\ensuremath{Ext}\xspace}

STAR-Vote keeps an electronic record of all the votes encrypted with a
threshold cryptosystem (so that decryption capabilities are
distributed to protect voter privacy) that has an additive homomorphic
property (to allow individual encrypted ballots to be combined into an
aggregate encryption of the tally).  The common exponential version of
the \elgamal cryptosystem satisfies the required properties, even
though stronger security is obtained by using PPATS
encryption~\cite{CPP13}, in particular against key manipulation errors
by the trustees and long-term security. The encryption scheme comes
with an extraction function \Ext that, from a ciphertext, extracts a
commitment on the encrypted value. In the case of \elgamal, this is be
the ciphertext itself, while in the case of PPATS, it is a perfectly
hiding homomorphic commitment.

Cryptographic key generation can be accomplished in one of two ways,
depending on the availability of the election trustees and the desired
amount of robustness. The preferred process offers general threshold
key generation requires multiple rounds~(see \cite{GJKR07} for
\elgamal and PPATS), but can be simplified into a single-round
solution if redundancy is eliminated.  At the end of the key
generation procedure, the trustees each hold a private key share that
does not contain any information on the full private key, and the
unique public key $K$ corresponding to those shares is published.

During the polling phase, the ballot marking devices encrypt the votes
of each voter using the public key $K$. This encryption procedure is
randomized in order to make sure that two votes for the same
candidates result in ciphertexts that look independent to any
observer. 

A cryptographic hash value of the commitment extracted from each
ciphertext (and of a few more data, as discussed below) is also
computed, fingerprinting the ballot to a 256 bit string, which is
provided to the voter in a human readable form as part of the
take-home receipt. All the hashes and commitments are computed and
posted on a publicly accessible web page, as soon as the polls are
closed. This web page is digitally signed by the election office using
a traditional signature key.

The posting of all the hashes gives all voters the ability to verify
that their ballots have been recorded properly.  The commitments that
can also be checked for consistency with the hashes and used to
confirm the homomorphic aggregation of the individual ballots into a
single encryption of the sum of the ballots, which constitutes an
encryption of the election tallies.

At the end of the election, any set of trustees that achieve the
pre-set quorum threshold use their respective private keys to decrypt
the derived aggregate tally encryption.  This procedure is simple and
efficient and can be completed locally without interaction between the
trustees.  We note that the individual encrypted ballots, from which
the aggregate encryption of the tallies is formed, are never
individually decrypted.  However, each spoiled ballot {\em is}
individually decrypted using exactly the same process that is used to
decrypt the aggregate tally encryption.

The elements we just described make the core of the workflow and are
sufficient to compute an election tally while preserving the privacy
of the votes. We now explain various ways in which this simple
workflow is hardened in order to make sure that the tally is also
correct. All the techniques that follow enable the verification of
different aspects of the ballot preparation and casting.

\paragraph{Hardening encryption}
\label{sec:hardening-encryption}
Since the tally does not involve the decryption of any individual
ballot, and since the audit procedure relies on the fact that all
counted ballots are properly formed, it is crucial to make sure that
all the encrypted ballots that are added correspond to valid votes.
This is achieved by requiring the ballot marking devices to compute,
together with the encryption of the votes, a non-interactive
zero-knowledge (NIZK) proof that each ballot is well-formed. Such a
proof guarantees that each ciphertext encrypts a valid vote and does
not leak any other information about the content of the vote. As a
side benefit, this proof can be designed to make the ballots
non-malleable, which provides an easy technique to prevent the replay
of old ballots (i.e., reject duplicates). Traditional sigma proofs
provide the required security properties~\cite{BPW12}.

\paragraph{Hardening decryption}
\label{sec:hardening-decryption}
Making sure that the encrypted ballots are valid is not enough: we
also need to make sure that the tally is correctly decrypted as a
function of those encrypted ballots: otherwise, malicious trustees (or
trustees using corrupted devices) could publish an outcome that does
not correspond to these ballots. As a result, we require the trustees
to provide evidence of the correctness of the decryption operations
that they perform.  This can also be accomplished with sigma
proofs in the case of \elgamal or more simply by publishing commitment
openings in the case of PPATS.

\paragraph{Hardening the timeline}
\label{sec:hard-timeline}

The procedures described above prevent malfunctioning or corrupted
voting terminals or trustees to falsify individual ballots or decryption
operations.

The detection of manipulation of encrypted ballots can be more
effective by linking ballots with each other, using hash chaining.
For this purpose, each ballot marking device is seeded, at the
beginning of the election, with a public start value $z_0$ that
includes a unique identifier for the election.  This unique identifier
is chosen at random shortly before the election, either in a central
way or by the poll workers themselves at the beginning of election
day.

% This unique identifier should be
% hard to predict and chosen as close as possible from the beginning of
% the polls in order to reduce the time frame during which a malicious
% party could prepare an attack.

From this seed, all elections event are chain hashed, with $z_{i+1}$
being computed as a hash of $z_i$ concatenated to the id of the
machine on which the event happens and to the event content. Two such
chains are maintained and properly separated. One is internal and
contains the full election data, including the encryption of the
votes, the casting time of each paper ballot, the information on
machines being added or removed,~\dots\ The second is public and
chains the commitment extracted from all encrypted votes, together
with time and identifiers for the election and voting machine. The
public hash is the one actually printed on the take-home receipt. When
the polls close, the final value of the hash chains are digitally
signed, and the public chain is made public together with all the
information needed for its reconstruction. 

As a result of this procedure, any removed ballot will
invalidate the hash chain which is committed to at the close of the
election and whose constituents appear on the voter take-home receipts.

\paragraph{Hardening the link between the paper and electronic
  election outcome}
\label{sec:hard-link-betw}

\newcommand{\bid}{\ensuremath{{bid}}\xspace}
\newcommand{\bcid}{\ensuremath{{bcid}}\xspace}

The voting terminals print human-readable versions of each ballot
summary which can be inspected for correctness by voters.  In addition
to the cast or challenge procedure discussed above, we need to produce
the data required for the risk-limiting audit described in
Section~\ref{sec:audit}.

To this purpose, we need to commit on a full electronic record
including a 1-to-1 mapping and evidence that this electronic record
leads to the announced outcome.
%
This is achieved as follows.
\begin{enumerate}
\item For each ballot, the ballot marking devices select a random ballot id
$\bid$. This \bid is printed on the ballots as a barcode. Furthermore,
for each race $r$ to which the voter participates, an encryption of
$H(\bid \| r)$ is also computed and appended to the encryption of the
choices.
\item At the end of the day, and before decryption of the tallies, the
  trustees (or their delegates) shuffle and rerandomize all encrypted
  votes, race by race. This shuffle does not need to be verifiable.
\item When the trustees decrypt the homomorphically added votes, they
  also decrypt the output of this shuffle. For each race, this
  provides a list of elements of the form $H(\bid \| r)$ and the
  corresponding cleartext choices.
\item Now, auditors can sample the paper ballots, read the \bid
  printed on them, recompute the value of $H(\bid \| r)$ for all races
  present on the paper ballot, and compare to the electronic record
  (as well as check many other things, as prescribed for the
  risk-limiting audit).
\end{enumerate}

The use of hashed \bid's has the important benefit of making sure that
someone who does not know a \bid value cannot, by looking at the
electronic record, link the selections made for the different races on
a single ballot, which protects from pattern voting attacks. There is
no need for such a protection from someone who can access the paper
ballots, since that person can already link all races just by looking
at the paper.

\paragraph{The full cryptographic protocol}
\label{sec:full-protocol}
The resulting cryptographic workflow is as follows. 
\begin{enumerate}
\item The trustees jointly generate a threshold public key/private key
  encryption pair. The encryption key $K$ is published.
\item Each voting terminal is initialized with the ballot/election
  parameters, the public key $K$ and seeds $z_0^p$ and $z_0^i$ that
  are computed by hashing all election parameters and a public random
  salt.
\item When a voter completes the ballot marking process selection
  to product a ballot $v$, the voting terminal performs the following operations: 
  \begin{enumerate}
  \item It selects a unique and unpredictable ballot identifier
    $\bid$, as well as a unique (but possibly predictable) ballot
    casting identifier \bcid.
  \item It computes an encryption $c_v = E_K(v)$ of the vote, as well
    as a NIZK proof $p_v$ that $c_v$ is an encryption of a valid
    ballot. This proof is written to be verifiable from $\Ext(c_v)$
    only.
  \item For each race $r_1, \dots, r_n$ to which the voter takes part, it computes
    an encryption $c_\bid = E_K(\bid \| r_1) \| \cdots \| E_K(\bid \| r_n)$. 
  \item It computes a public hash code $z^p_i = H(\bcid \| \Ext(c_v) \| p_v \|
    m \| z^p_{i-1})$, where $m$ is the voting terminal unique
    identifier, as well as an internal hash $z^i_i = H(\bcid \| c_v \| p_v \|
    c_\bid \| \| m \| z_{i-1}^i)$
  \item It prints a paper ballot in two parts. The first contains $v$
    in a human readable format as well as $c_\bid$ and $\bcid$ in a
    robust machine readable format (e.g., as barcodes). The second is
    a voter take-home receipt that includes, the voting terminal
    identifier $m$, the date and time, and the hash code $z^p_i$ (or a
    truncation thereof), all in a human-readable format.
  \item It transmits $(\bcid, c_v, p_v, c_\bid, m, z^p_i, z^i_i)$ to
    the judge's station.
  \end{enumerate}
\item When a ballot is cast, the ballot casting id $\bcid$ is scanned
  and sent to the judge's station.  The judge's station then marks the
  associated ballot as cast and ready to be included in the
  tally. This information is also broadcast and added in the two hash
  chains.
\item When the polls are closed, the tally is computed: the product of
  all cast encrypted votes is computed and verifiably decrypted,
  providing an election result.
\item The data needed for the risk limiting audit is computed, as described above. 
\end{enumerate}

All the data included in the public hash chain are eventually
digitally signed and published by the local authority. Those audit
data are considered to be valid if the hash chain checks, if all
cryptographic proofs check, that is, if the ballot validity proofs
check, if the homomorphic aggregation of the committed votes is
computed and opened correctly, and if all spoiled ballots are
decrypted correctly.

\paragraph{Write-in votes}
So far, we have not described how our cryptographic construction can
support write-in voting. Support for write-in votes is required in
Texas and many other states. To be general-purpose, \projname adopts
the vector-ballot approach~\cite{kiayias04vectorBallot}, wherein there
is a separate homomorphic counter for the write-in slot plus an
encryption of the string in the write-in. If there are enough write-in
votes to influence the election outcome, then the write-in slots,
across the whole election, will be mixed and tallied (together with
the corresponding counters).

We note that, at least in Texas, write-in candidates must be
registered in advance. It's conceivable that we could simply allocate
a separate homomorphic counter for each registered candidate and have
the \projname terminal help the voter select the desired ``write-in''
candidate. Such an approach could have significant usability benefits
but is expected to require some update of regulations.

% \paragraph{Summing up} The elements described above lead to the
% following workflow. \COP{Sum up, explaining how all elements are
%   computed and in what order.}
% \begin{compactenum}
% \item The trustees choose a threshold $t$ and run a the corresponding
%   key generation procedure. At the end of this procedure, each trustee
%   holds a private key and a unique public key $K$ is published.
% \item At the beginning of the election day, a public start value $z_0$
%   is chosen that uniquely identifies the election and voting
%   office. All ballot marking devices are initialized with the election
%   specifics (ballot format, \dots), with the key $K$ and with the seed $z_0$.
% \item After the completion of the selection by each voter, the ballot
%   marking device proceeds as follows.
%   \begin{compactenum}[a.]
%   \item It computes an encryption $b_i$ of the voter's selection using
%     the public key $K$, together with a ballot validity proof
%     $pb_i$. These elements are hashed in order to produce a ballot
%     fingerprint $f_i = H(b_i, pb_i, z_{i-1}, m)$.
%   \item It produces paper printouts of two items. The first is the
%     paper ballot which consists of the selections made by the voter
%     and $f_i$. The second is a receipt that consists of an
%     identification number $m$ for the ballot marking device, the date
%     and time of the vote, and a truncated SHA-256 hash of the \elgamal
%     encryption of the voter’s selections together with the previous
%     hash value.

%   \end{compactenum}
% \end{compactenum}


%%% Local Variables: 
%%% mode: latex
%%% TeX-master: "star"
%%% End: 
