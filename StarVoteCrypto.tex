\documentclass[prodmode]{acmsmall}

\usepackage{xspace}
\usepackage{url}
\usepackage{amssymb}

\title{STAR-Vote\\ Cryptographic Aspects}
\author{Josh Benaloh\affil{Microsoft Research} \and Olivier Pereira\affil{Universit{\'e} catholique de Louvain}}

\newcommand{\COP}[1]{{\sf [[OP: #1]]}}

% Notations

% Public keys
\newcommand{\kt}{\ensuremath{k_T}\xspace}
\newcommand{\ka}{\ensuremath{k_A}\xspace}
\newcommand{\kal}{\ensuremath{k_{AL}}\xspace}
\newcommand{\kc}{\ensuremath{k_C}\xspace}

% Hash chain seeds
\newcommand{\zp}{\ensuremath{zp}\xspace}
\newcommand{\zi}{\ensuremath{zi}\xspace}

% Identifiers
\newcommand{\bst}{\ensuremath{bst}\xspace}   % Ballot style
\newcommand{\bid}{\ensuremath{bid}\xspace}   % Ballot id
\newcommand{\bcid}{\ensuremath{bcid}\xspace} % Ballot casting id
\newcommand{\uid}{\ensuremath{uid}\xspace}   % Urn id
\newcommand{\cid}{\ensuremath{cid}\xspace}   % Controller id
\newcommand{\bmd}{\textsf{BMD}\xspace}       % BMD
\newcommand{\bmdid}{\ensuremath{mid}\xspace} % BMD id
\newcommand{\token}{\ensuremath{t}\xspace}   % token

% Cryptographic functions
\newcommand{\Enc}[2]{\textsf{Enc}_{#1}(#2)\xspace} % Encryption
\newcommand{\CCEnc}[2]{\textsf{CCEnc}_{#1}(#2)\xspace} % Commitment Consistent Encryption
\newcommand{\CCExt}[1]{\textsf{CCExt}(#1)\xspace} % Commitment Extraction
\newcommand{\hash}{\ensuremath{\mathcal{H}}\xspace} % Hash function 

% Variables
\newcommand{\vote}{\ensuremath{v}\xspace} % Vote
\newcommand{\cv}{\ensuremath{c_v}\xspace} % Encryption of vote
\newcommand{\cpv}{\ensuremath{c'_v}\xspace} % Shuffled encryption of vote
\newcommand{\cpva}{\ensuremath{c^{'A}_v}\xspace} % Re-encrypted shuffled votes
\newcommand{\pcv}{\ensuremath{\mathbf{c_v}}\xspace} % Encryption of the product of the votes
\newcommand{\cb}{\ensuremath{c_{\bid}}\xspace} % Encryption of ballot id for auditors
\newcommand{\cpb}{\ensuremath{c_{\bid}}\xspace} % Shuffled encryption of ballot id for auditors
\newcommand{\cbc}{\ensuremath{c^C_{\bcid}}\xspace} % Encryption of ballot casting id for controller
\newcommand{\cpbc}{\ensuremath{c^{'C}_{\bcid}}\xspace} % Encryption of ballot casting id for controller

\newcommand{\G}{\ensuremath{\mathbb{G}}\xspace} % Group



\begin{document}
\maketitle
\section{Election Process}
\label{sec:election-process}

\subsection{Entities}

We have parties assuming 3 different roles:
\begin{description}
\item[Voters] submit votes and want to be able to perform various audit operations. 
\item[Trustees] compute the election outcome and are responsible
  of the privacy of the votes.
\item[Internal Auditors] are responsible of doing internal audit
  operations. This in particular includes running a risk-limiting
  audit.
\end{description}

We also have devices assuming 4 different roles:
\begin{description}
\item[Ballot Marking Devices] interact with the voters in order
  to produce ballots, under electronic and paper format.
\item[Urns] receive the paper ballots and notify of this
  reception.
\item[Auditing Stations] can be used by voter in order to challenge
  ballot marking devices.
\item[The Controller] orchestrates the various devices in a voting
  office. 
\end{description}


\subsection{Setup}
\label{sec:setup}


The following values are created an published before the election
starts.
\begin{enumerate}
\item Trustees produce an election public key \kt for a threshold
  commitment consistent encryption scheme~\cite{CPP13}.
\item The controller produces a public key \kc for an encryption
  scheme.
\item Two unique hash chain seeds $\zp_0$ and $\zi_0$, one for the
  public audit part and one for the internal audit part. These seeds
  are computed as cryptographic hashes of a string representing in a
  unique, unambiguous and unpredictable way the description of the
  election, to which is concatenated a special string indicating the
  chain in which that seed must be
  used. 
% This includes the election name, date,
  % the description of all races, ballot formats, \dots The inputs of
  % the hash function also include an unpredictable public salt chosen
  % shortly (a few days) before the beginning of the election in
  % order to limit the amount of time that attackers might have to
  % precompute specific collisions on the hash function.
\end{enumerate}
The unpredictability may result from poll workers jointly rolling
dices on the election morning in each polling station and entering the
outcome of the dice rolls into the controller. 

All ballot marking devices are initialized with \kt, \kc,
$\zp_0$, $\zi_0$. The urns are initialized with \kc, $\zp_0$ and
$\zi_0$.




\subsection{Voter identification}
\label{sec:voter-identification}
When signing-in, a voter receives a token \token associated to his
ballot style \bst. This token is made in such a way that it can be
used only once. 
%
Recovery procedures are decided in the case a voter
claims that his token was consumed by a \bmd that did not return a
paper ballot.




\subsection{At the Ballot Marking Device}
\label{sec:at-ballot-marking}

When entering the booth, the voter enters his token and the \bmd
displays an empty ballot of style \bst, so that the voter can make his
selection $\vote$.

% When entering the booth, the voter presses a ``Start'' button on the Ballot Marking Device (BMD). That makes the BMD print a unique random ballot identification number (BID) in such a way that the voter is able to tear up that part of the ballot if the BMD stops working, as an evidence that he did 

% \COP{How do we prevent a cheating machine for randomly refusing to print ballots that go in one direction and being able to detect that behavior against the behavior of a voter who would like to get a chance to cast two ballots?}

The \bmd processes these choices as follows: 
\begin{itemize}
\item It broadcasts the information that the token introduced by the
  user is consumed.
\item It selects an unpredictable ballot identifier $\bid$ (e.g., as a
  UUID.) This \bid must be kept secret and will be part of the risk
  limiting audit trail.
\item It selects a unique ballot casting identifier \bcid
  (possibly as the concatenation of the \bmdid and a counter). This
  \bcid will be used to track whether the ballot made its way to the urn. 
\item It computes $\cbc = \Enc{\kc}{\bcid}$, which will be 
  used by the controller to link the paper and electronic ballot.
\item It computes $\cv = \CCEnc{\kt}{\vote, \zp_0}$ and a vector $\cb$
  of ciphertexts that encrypt $\hash(\bid \| r_i)$ with $\kt$ for each
  race $r_i$. 
\item It broadcasts a message that contains the identifier of the
  ballot marking device $\bmdid$, the time $t$, $\bst$, $\cbc$,
  $\cb$, $\cv$ and a hash $\zi_i := \hash(\zi_{i-1}, \bmdid, t,
  \bst, \linebreak[4] \cbc, \cb, \cv)$ for inclusion in the internal audit trail.
\item It prints a ballot made of two pieces that can be easily taken
  apart. The first part, contains a human readable summary of \vote
  and a machine readable version (as a 1D barcode) of $\bid$, $\bcid$
  and the ballot style $\bst$ (if the ballot is made of multiple
  pages, then the page number is concatenated to the $\bst$). The
  second part is a take home receipt that contains a human readable
  version of the election description, the time $t$ and $\zp_i :=
  \hash(\zp_{i-1}, \bmdid, t, \CCExt{\cv})$. In order to
  simplify the audit process, the first 5 symbols of $\zp_i$ are
  printed in bold and a bit separated from the other symbols. At
  verification time, the voter will then be invited to search for his
  receipt based on these 5 characters and to check whether the reply
  contains the rest of the hash.
\end{itemize}

When receiving this, the controller decrypts $\cbc$ and appends the
pair $(\bcid, \zp_i)$ into a local table, until a ballot with that \bcid
is scanned at the urn. Note that this table should not appear in any
log, as being able to link a \bcid to a specific time and that time to
the time at which a voter submitted a specific ballot could violate
the vote privacy during the internal audit process. In case of
problem, the encrypted version of the \bcid still appears in all logs,
which should provide the information needed if an investigation is needed.

\subsection{Challenging the Ballot Marking Device}
\label{sec:chall-ball-mark}

If the voter wants to challenge the \bmd, it brings the printed ballot
to a pollworker. The pollworker now proceeds as follows: 
\begin{itemize}
\item He physically stamps the ballot to mark it as spoiled, and links
  its different parts together in such a way that noone will be able
  to rebuild a full ballot from parts of different ballots in an
  unnoticeable way (possibly by stamping a unique serial number on all
  parts of the ballot). This makes that ballot usable as evidence in
  case of a cheating \bmd. Using specially designed paper would be
  desirable as well, in order to make forgeries harder. 
\item He scans the $\bcid$ so that the ballot is recorded to be tallied
  as part of the spoiled ballots urn. This recording process happens
  in the same way as a normal casting process, except that a
  ``spoiled'' flag appears in all messages included in the hash
  chain. (This scanning step is optional.  Its goal is to facilitate,
  as a human factor verification, the counting of the number of
  spoiled ballots against the number of voters who leave the voting
  office without casting their ballot.)
\item He gives the ballot back to the voter. 
\end{itemize}

Later, at tallying time, the spoiled ballots will all be decrypted and
posted on a bulletin board for voter verification. Any voter who
either sees that his ballot does not appear decrypted on the board, or
that the published decryption does not match the human readable
choices on his ballot, is invited to complain. 

\COP{Talk about how to deal with the complain.}

\subsection{Casting a Ballot}
\label{sec:casting-ballot}

If the voter is happy with the ballot printed by the \bmd, it brings
it to the urn. There, the two pieces of the ballot are split, the part
containing the machine-readable \bcid is scanned and placed into the
urn, while the other part containing the take-home receipt is kept by
the voter. 

The urn computes $\cpbc = \Enc{\kc}{\bcid}$ and sends it to the
controller, together with $\zi_i = \hash(\zi_{i-1}, t, \uid,
\cpbc)$. The controller decrypts \cpbc and search for \bcid in its
local ballot table. If \bcid is not in the table, then it triggers an
alert. Otherwise, if it finds a $(\bid, z)$ pair in its table, it
removes it from the table, broadcasts $z$ and $\zi_i =
\hash{H}(\zi_{i-1}, t, \cid, z)$, which indicates that the ballot
that is associated to $z$ in the hash chain must be included in the
tally.

\subsection{Tallying process}
\label{sec:tallying-process}

At the end of the day, all $\cv$'s that have been marked as to be
included for the tally are checked for validity and aggregated into an
encryption $\pcv$ of the tally. This tally is then jointly decrypted
by the trustees and published. (This is done as needed for the
different races, ballot styles, \dots)

Then, $\CCExt$ is applied to all $\cv$ ciphertexts and the result is
published with all the information needed to check the $\zp$ hash
chain, and the trustees publish the information that is needed to
prove that the tally is consistent with the $\CCExt{\cv}$'s.

Besides, the trustees also jointly verifiably decrypt and publish the
content of all spoiled ballots. 

\subsection{Audit of the Electronic Process}
\label{sec:publ-audit-electr}

Anyone can perform a number of verifications from the published information: 
\begin{enumerate}
\item check the validity of the published $\CCExt{\cv}$'s;
\item check that the tally is consistent with the published $\CCExt{\cv}$'s;
\item check the validity of the $\zp$ hash chain; 
\item check the number of ballots against the number of voters if the
  information is public.
\end{enumerate}

Furthermore, voters are invited to check whether the $\zp_i$ value
printed on their receipt appears in the list of ballots included in
the tally. 

All published information is available through an easy to access web
API in order to ease the design of independent auditing tools, but
also as a big signed document that commits the election organizers on
the posted values.

If any of these verification fails, people should complain.

\COP{Discuss how to deal with these complains. }

\subsection{Audit of the Paper Ballots}
\label{sec:risk-limiting-audit}

After having checked the validity of all the ballots, but before
decrypting the election outcome, the trustees supervise (or perform),
contest by contest, a shuffle of all $(\cb, \cv)$ pairs corresponding
to valid ballots, yielding a list of $(\cpb, \cpv)$ pairs. This
shuffle is performed in an optimistic way and does not need to be
verifiable. As a result, it can be performed extremely efficiently:
all reencryption ciphertexts can be precomputed as $\Enc{\kt}{0}$,
leaving only a bunch of multiplications to be performed at tallying
time.

After completion of this shuffle, the trustees decrypt all \cpv and
$\cpb$ tuples, at the same time as they decrypt the election
outcome. This decryption yields, for each race $r_i$, a list that
contains $\hash(bid\|r_i)$ and the cleartext choices that should be
those on the paper ballot $\bid$ for race $r_i$. This table is made
available to all people and observers who take part to the
risk-limiting audit. Note that, thanks to the use of the hash
function, noone is able to decide which race results belong to the
same ballot: this can only be recovered by someone who knows the \bid,
which can be learned from the paper ballots. But, in that case, the
full ballot is exposed anyway.

It may be desirable to publicly commit to that table. However, the
$\hash(bid\|r_i)$ should then be removed as a voter who would have
recorded his \bid could use it to prove that he voted in some
way. This is not a problem regarding insiders, as it does not
introduce any weakening of the privacy of the votes with respect to
them.

From this table, a risk limiting audit can take place. One way of
doing it is as follows. Make sure that all hashes are unique and that
the table contains as many races as we should have from the paper
ballots, then repeat as long as decided for the audit:
\begin{enumerate}
\item verifiably select a random paper ballot in the urn;
\item read its \bid and search for $\hash(\bid\|r_i)$ values in the
  table for all races $r_i$ present on the ballot;
\item compare the corresponding plaintexts to the paper ballot.
\end{enumerate}

Proper care should also be taken to make sure that it is not feasible
to inspect all paper ballots in search of a specific pattern.%  It seems
% much easier to have this kind of control on paper ballots than on an
% electronic record on a computer.

\section{Cryptographic algorithms}

For practical use, we target using:
\begin{itemize}
\item Encryption schemes that come with a simple distributed/threshold
  key generation procedure, in order to simplify and limit the risks
  of error. In this context, encryption schemes that allow computing
  in prime order groups are of special interest. 
\item A security level equivalent to at least 256-bits. This applies
  for the output size of hash functions (e.g., SHA256 or Keccak) and
  the order of the groups in which we would compute (see
  \cite{keylength} for instance.)
\end{itemize}

We discuss some concrete proposals below. 

\subsection{Key generation}
\label{sec:key-generation}

We use groups with a public generator of prime order $q$. In all the
schemes we discuss, the public key is a sequence of elements of the
form $g^x$ and the private key is a sequence of corresponding $x$,
possibly shared between a set of key holders.

This can be done in a very simple way in the case of distributed
(non-threshold) key generation. Let $U_1, \dots, U_n$ be the set of
key holders and $l$ be a unique election identifier. $U_i$ proceeds as
follows:
\begin{enumerate}
\item select $(x_i, r_i) \leftarrow \mathbb{Z}^2_q$;
\item publish $g^{x_i}$, and the Schnorr proof made of $c_i=
  \hash(l, g^{x_i}, g^{r_i})$ and $f_i= r_i + c_ix_i$.
\end{enumerate}
Everyone can now verify that $c_i = \hash(l, g^{x_i},
g^{f_i}/(g^{x_i})^{c_i})$ and compute the public key element as
$g^x = \prod_i g^{x_i}$.

The key benefit of this approach is that it requires a single round by
the key holders and does not require any private communication between
key holders. A single key loss is however sufficient to make it
infeasible to decrypt any ciphertext.
%
Various threshold key generation procedures exist but are much more
demanding~\cite{GJKR07}.

\COP{Helios uses distributed key generation, which is already often
  complicated to organize (trustees availability, understanding of the
  procedure, \dots) especially as it is important to minimize the
  supervision at that point for privacy reasons. But there is a clear
  fear of losing a key at some point as well. This is often mitigated
  by pairing the trustees and then having at least two copies of each
  key. It would be an interesting effort to see how to organize actual
  threshold key generation in a usable way.}

We want a commitment consistent encryption scheme for the trustees,
that makes it simple to have verifiable encryption and decryption
procedures. 

\subsection{Encryption of 0/1 values}
\label{sec:encryption-trustees}

The scheme must also be additively homomorphic and it should be
feasible to decrypt the sum of all votes efficiently.

The most effective way to do this seems to encrypt all choices
independently and, for validity, to prove that each ciphertext
encrypts a 0 or a 1 (assuming approval voting). All unregistered
write-ins can be aggregated in a single candidate, which prevents
having a full count for them individually. We hope that the number of
votes they will get will be small enough to not require decrypting
write-in ciphertexts in the end. 

We describe here two ways to verifiably encrypt and decrypt a 0 or a
1. The first involves simpler algebraic operations (basic modular
arithmetic is sufficient) but is less robust from a confidentiality
point of view. The second offers stronger resilience to mistakes by
trustees, hacking and advances in computational power and
cryptanalysis, but involves more sophisticated algebraic operations
(elliptic curve point operations for voting and pairing evaluation for
auditing.)

\subsubsection{ElGamal encryption}
\label{sec:elgamal-encryption}

The simplest choice for encryption is to use exponential ElGamal. 
\begin{itemize}
\item The public key is a single group element $h = g^x$ generated as
suggested above, in a distributed or threshold way.
\item A choice $m$ is encrypted as follows: choose $r \leftarrow
\mathbb{Z}_q$ and compute $(\alpha, \beta) = (g^r, g^mh^r)$.
\item The fact that a ciphertext encrypts 0 or 1 is proven by adding
the following disjunctive Chaum-Pedersen proof: select $c_{1-m}
\leftarrow \mathbb{Z}_{2^{256}}$ and $f_{1-m} \leftarrow
\mathbb{Z}_q$, compute $A_{1-m} = g^{f_{1-m}}/(\alpha)^{c_{1-m}}$ and
$B_{1-m} = h^{f_{1-m}}/(\beta/g^{1-m})^{c_{1-m}}$, select $s
\leftarrow \mathbb{Z}_q$, compute $A_m = g^s$ and $B_m = h^s$, then $c
= \hash(\zp_0, \alpha, \beta, A_0, B_0, A_1, B_1)$. Eventually,
compute $c_m = c - c_{1-m}$ and $f_m = s + c_mr$. The proof is made of
$c_0, c_1, f_0, f_1$.
\item Decryption proceeds by computing a decryption factor $\delta =
\alpha^x$. Then, $m$ can be computed as the discrete logarithm of
$\beta/\delta$ in basis $g$.
\item The correctness of the decryption can be proven by adding the
following proof. Select $s \leftarrow \mathbb{Z}_q$, compute and
publish $c = \hash(\zp_0, \alpha, \delta, g^s, h^s)$ and $f = s
+ cx$.
\end{itemize}

Several observations here: 
\begin{description}
\item[Group choice] The group used here can either be a 256 bits prime
  order subgroup of $\mathbb{Z}_p^*$ where $|p| = 2048$ (or even 3248
  for general long-term protection) or a 256 bits prime order group on
  an elliptic curve. Actual choices for these groups are proposed in
  various places (e.g., \cite{rfc5114,dss}). Explicit formula for
  elliptic curve operations are widely available as
  well~\cite{hyperelliptic}. Elliptic curves require more
  sophisticated algebra but offer ciphertexts that are typically
  $\approx$ 8--10 times smaller for the same security level.
\item[Discrete log extraction] We use exponential ElGamal in order to
  have an additively homomorphic encryption scheme. As a result, the
  decryption procedure involves extracting a discrete logarithm (DL)
  in a fixed base. As the number of voters is an upper bound on this
  DL, an exhaustive search is easy. The full list of values can also
  be stored in a table in advance if desired. A middle ground consists
  in using Shanks' baby-step giant-step algorithm. 
\item[Distributed decryption] Distributed decryption can be done as
  follows: each key holder computes $\delta_i = \alpha^{x_i}$ and now
  $m$ is the DL of $\beta/\prod_i \delta_i$ in base $g$. Each key
  holder also needs to provide a Chaum-Pedersen proof of correct
  decryption. 
\item[\textsf{CCExt}] The \textsf{CCExt} function is defined here as
  the identity function. As a result, when using ElGamal, the full
  ciphertexts are included as part of the public audit trail \zp. This
  may result in a full loss of the privacy of the votes if the
  trustees somehow lose their keys, if a cryptanalytic breakthrough
  happens, or if people look at the records in a few dozen of years
  from now when computational power will have increased enough to
  break the encryption easily.
\end{description}

\subsubsection{PPATS encryption}
\label{sec:ppats-encryption}

The PPATS scheme~\cite{CPP13} makes it possible to have public audit trail
perfectly hiding the votes, at the potential cost of more
sophisticated algebraic operations. By using this scheme, the public
audit trail does not risk helping to violate the privacy of the votes
if keys are revealed or encryption broken.

This scheme uses 3 groups of the same prime order $q$: $\G_1$, $\G_2$
and $\G_T$, in a setting where there is an asymmetric bilinear map $e:
\G_1 \times \G_2 \rightarrow \G_T$. We use so-called Type-3
groups~\cite{GPS08}, so that the DDH problem on which the security of
ElGamal relies is expected to be hard in these groups. We suppose that
$g_1$, $g_2$ and $g_T$ generate $\G_1$, $\G_2$ and $\G_T$
respectively. We furthermore assume that $h_2$ is a second generator
of $\G_2$ chosen randomly and independent of $g_2$ (in practice, this
can be done by computing $h_2$ deterministically as a cryptographic
hash of $g_2$ for instance).

\begin{itemize}
\item The public key is a single group element $h_1 = g_1^x$
  generated as suggested above, in a distributed or threshold way.
\item A choice $m$ is encrypted as follows: choose $r, s \leftarrow
  \mathbb{Z}^2_q$ and compute $(\alpha, \beta, \gamma) = (g_1^s,
  g_1^rh_1^s, h_2^mg_2^r)$.

\item The fact that a ciphertext encrypts 0 or 1 is proven by adding
the following proofs: 
\begin{itemize}
\item Choose $m_c, r_c, s_c \leftarrow \mathbb{Z}^3_q$ and compute
  $(A_c, B_c, C_c) = (g_1^{s_c}, g_1^{r_c}h_1^{s_c},
  h_2^{m_c}g_2^{r_c})$. The proof is made of $c_c = \hash(\zp_0, \alpha, \beta,
  \gamma, A_c, B_c, C_c)$ and $f_c = (m_c + c_cm, r_c + c_cr, s_c +
  c_cs)$;
\item select $c_{1-m} \leftarrow \mathbb{Z}_{2^{256}}$ and $f_{1-m}
  \leftarrow \mathbb{Z}_q$, compute $C_{1-m} =
  g_2^{f_{1-m}}/(\gamma/h_2^{1-m})^{c_{1-m}}$ select $r_o \leftarrow
  \mathbb{Z}_q$, compute $C_m = g_2^{r_o}$, then $c =
  \hash(\zp_0, \gamma, C_0, C_1)$. Eventually, compute $c_m = c
  - c_{1-m}$ and $f_m = r_o + c_mr$. The proof is made of $c_0, c_1,
  f_0, f_1$.
\end{itemize}

\item $\CCExt{\alpha, \beta, \gamma, c_c, f_c, c_0, c_1, f_0, f_1} =
  (\gamma, c_0, c_1, f_0, f_1)$.

\item Decryption proceeds by computing a decryption factor $\delta =
\alpha^x$. Then, $m$ can be computed as the discrete logarithm of
$e(\delta/\beta, g_2) \cdot e(g_1, \gamma)$ in basis $e(g_1,h_2)$.
\item The correctness of the decryption can be proven by publishing
  $a = \alpha^x$. (It can then be checked that $e(\beta/a, g_2) = e(g_1, \gamma/h_2^m)$.) 
\end{itemize}

Several observations here: 

\begin{description}
\item[Group choice] The most common curve choices offering the
  properties we need are the BLS curves~\cite{BLS02,CLN11} or the BN
  curves~\cite{BN05,PSNB11}. Various free implementations of the BN
  curve arithmetic are
  available~\cite{Naehrig,PBC,bnpairings}. \COP{Microsoft Redmond
    researchers wrote very efficient software implementations for the
    these curves.}
\item[\textsf{CCExt}] The \textsf{CCExt} function now extracts a
  single group element from the ciphertext, which perfectly hides the
  vote, as well as the 0-1 validity proof, which is perfectly hiding
  as well. So, whatever cryptography breakthrough or hacking of the
  trustees happens, the output of \textsf{CCExt} will not leak
  anything about the votes.
\end{description}

\subsection{Encryption of the \bid}
\label{sec:encrypt-large}

We need to encrypt larger values in various places and, most
importantly, the $\bid$ which is a high entropy value, and needs to be
homomorphically encrypted. The high entropy prevents using a scheme in
exponential mode: we would not be able to extract the final DL at
decryption time.

Here are two ways to solve this: 
\begin{enumerate}
\item Use ElGamal on elliptic curves (not in $\mathbb{Z}_p^*$): it is
  then easy to encode and decode integers as group elements.
\item Generate a second ElGamal key in a different group
  $\mathbb{Z}_p^*$ with $p=2q+1$ where $q$ is a 2047 bits prime and
  $p$ is equal to $3 \bmod 4$. Then, squaring messages $\bmod p$ is an
  easily decodable encoding of messages. Note that we do not want to
  use this group everywhere, as encryption becomes 8 times more expensive.
\end{enumerate}

This suggests that elliptic curve cryptography might be more
convenient by limiting the number of keys.

% A simple way to solve this problem is to work is the
% subgroup of order $q$ of a 2048 bits prime $p=2q+1$, where $p=3 \bmod
% 4$. In that case, the square root extraction is easy and, if $1\leq m
% \leq q$, then $m^2$ is an element of the group and the encryption can
% be computed as $(g^r, m^2h^r)$. So, $\ka$ will be a key in a group of
% 256 bits order, while $\kal$ will be a key in a group of 2047 bits
% order.

\subsection{Encryption for the controller}
\label{sec:encrypt-contr}

The encryption scheme used to send the \bcid to the controller does not
need any homomorphic property but needs to support efficient decryption.

The solutions above can be used again here, as well as more
traditional solutions if that eases the implementation (e.g., RSA
encryption, or hashed ElGamal).

% \subsection{Verifiable shuffle}
% \label{sec:verifiable-shuffle}

% Most verifiable shuffles can be used for our purpose. The approach of
% Terelius and Wikstr{\"o}m~\cite{Wik09,TW10} has been successfully used
% in Helios~\cite{BGP11} and Wombat~\cite{wombat}.

% Note that we actually need the shuffle (and reencryption) to be
% verifiable in order to make sure that the list of ciphertexts that is
% used for the risk-limiting audit contains only encryptions of 0 and 1
% (and that they match the election outcome).

% \subsection{Reencryption for the auditors}
% \label{sec:reencrypt-audit}

% The output of the mixnet has a set of ciphertexts that are encrypted
% with the $\kt$ key. The trustees need to transform these into
% ciphertexts encrypted with the \ka key. We show how to do this in the
% case of distributed (non threshold) keys.


% \begin{enumerate}
% \item In the ElGamal case, the decryption factor is replaced by an
%   encryption of it using \ka, that is, given a ciphertext $(\alpha,
%   \beta)$, the trustee $U_i$ selects $r_i \leftarrow \mathbb{Z}_q$ and
%   computes $(\alpha'_i, \beta'_i) = (g^{r_i},
%   \ka^{r_i}/\alpha^{x_i})$. Eventually, the ciphertext $(\alpha',
%   \beta')$ is computed as $(\prod_i \alpha'_i, \beta \cdot \prod_i
%   \beta_i) = (g^{\sum_i r_i}, m\cdot \ka^{\sum_i r_i})$. Each trustee
%   also proofs that this transform was performed correctly by selecting
%   $(s_i, y_i) \leftarrow \mathbb{Z}_q$, then $c_i = \hash(\zi_0,
%   \alpha, g^{r_i}, g^{s_i}, \ka^{s_i}/\alpha^{y_i})$ and $(f_1, f_2) =
%   (s_i + c_i r_i, y_i + c_i x_i)$.
% \item In the PPAT1 case, the procedure is the same: just consider the
%   first two elements $(\alpha, \beta)$ as an ElGamal ciphertext.
% \end{enumerate}

% Note that, in both cases, this requires using the same groups and
% encryption schemes for encryption with \kt and \ka.



\bibliographystyle{plain}
\bibliography{StarVoteCrypto}


\end{document}
